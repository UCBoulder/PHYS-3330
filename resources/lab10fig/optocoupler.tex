% Process for creating PNG:
% 1 - Compile this document
% 2 - Download PDF
% 3 - Upload to https://cloudconvert.com/pdf-to-png
% 4 - Set Pixel Density to 600
% 5 - Download PNG file and upload to Overleaf

\documentclass[border=0.2cm]{standalone}
\usepackage{circuitikz}
\usepackage{amsmath}



\standaloneenv{circuitikz}
\begin{document}


\begin{circuitikz}[american voltages]
\ctikzset{bipoles/diode/height=.3, bipoles/diode/width=.3,}
    
    \draw (0,0) node[npn](optonpn) {} coordinate(okay)
    
    (optonpn.emitter) to [short,-o] ++(1,0) to ++(0,0) node[anchor=west] {emitter}

    (optonpn.collector) to [short,-o] ++(1,0) to ++(0,0) node[anchor=west] {collector};
    
    \fill[white] (okay) ++(-1,-0.25) rectangle ++(.566, .5);

    \draw (okay) ++(-3, .75) node[anchor=east] {Arduino GPIO} to [short,-o] ++(0,0) to ++(2,0) to [empty led]  ++(0,-1.5) to ++(-0.75,0) to [R,l_=$R_{cl}$] ++(-2.5,0) to ++(0,0) node[sground]{}
    ;
    \draw (okay) ++(-3.5, -1.75) node {Arduino common ground};

    \draw[thick] (okay) ++(-1.5, 1) node {} to ++(1.8,0) to ++(0,-2) to ++(-1.8,0) to ++(0,2);

    \draw (okay) ++(-.55,1.25) node {optocoupler};

    
    
    
\end{circuitikz}



\end{document}


 