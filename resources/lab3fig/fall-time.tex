% Process for creating PNG:
% 1 - Compile this document
% 2 - Download PDF
% 3 - Upload to https://cloudconvert.com/pdf-to-png
% 4 - Set Pixel Density to 600
% 5 - Download PNG file and upload to Overleaf

\documentclass[border=0.2cm]{standalone}
\usepackage{circuitikz}
\usepackage{amsmath}

\standaloneenv{circuitikz}
\begin{document}


\begin{circuitikz}[american voltages]
    \draw (.1,1.5) to ++(-.25,0) node[anchor=east, scale=0.75]{$V_0$};

    \draw[thick,->] (0,0) -- (6,0) node[anchor=west]{$t$};
    \draw[thick,->,cap=rect] (0,0) -- (0,2) node[anchor=south]{$V$};
    \draw[line width=2, cap=round, color=blue](0,0) to ++(0,1.5) to ++(2.5,0) to ++(0,-1.5) to ++(2.5,0) to ++(0,1.5) to ++(1,0)
    
    (5,2.5) to ++(0.25,0) node[color=black,anchor=west]{$V_\text{in}$};
    
    \draw[line width=2, cap=round, color=red](5,2) to ++(0.25,0) node[color=black,anchor=west]{$V_\text{out}$};
    
    \draw[smooth, cap=round, color=red, domain=0:2.5, line width=2] plot (\x,{1.5*(1-exp(-2*\x))});

    \draw[smooth, color=red, domain=2.5:5, line width=2] plot (\x,{1.5*exp(-2*(\x-2.5))});

    \draw[smooth, cap=round, color=red, domain=5:6, line width=2] plot (\x,{1.5*(1-exp(-2*(\x-5)))});

    \draw[<-, smooth] (3,1) -- (3.2,1.2) -- (3.2,2) node[anchor=south]{$V_0e^{-t/\tau}$};

    
    

    
    
\end{circuitikz}

\end{document}

